%% NOTICE: Overleaf free plan is limiting this compilation, so it is better to compile using your own computer
\documentclass{beamer}

%% TODO: Uncomment the next three lines to add notes to presentation slides
%\usepackage{pgfpages}
%\setbeameroption{show notes on second screen}
%\setbeamertemplate{note page}{\insertnote}

\usepackage[utf8]{inputenc}

\usepackage{graphicx}

%% TODO: Put all the figure files inside the images folder
\graphicspath{{images/}}

\usepackage{amsmath}
\usepackage{amssymb}

\usepackage{pifont}

%% TODO: Use \cmark for tick and \xmark for x
\newcommand{\cmark}{\ding{51}}%
\newcommand{\xmark}{\ding{55}}%

\usepackage{algorithm}
\usepackage{algpseudocode}

%% TODO: Comment the next three lines to remove the bibliography
\usepackage[backend=biber,style=numeric, citestyle=ieee]{biblatex}
\addbibresource{References.bib}  % Fixed: Changed from bibliography.bib to References.bib
\AtBeginBibliography{\small}

\usepackage{appendixnumberbeamer}
\pdfstringdefDisableCommands{%
\def\translate#1{#1}%
}

%% TODO: Comment the next line if you need the control symbol
\beamertemplatenavigationsymbolsempty

\usetheme{Berlin}
\useinnertheme{circles}

\AtBeginSection[]
{
 \begin{frame}
 \frametitle{Table of Contents}
 \tableofcontents[currentsection]
 \end{frame}
}

%% TODO: Add information to the title page
\title{Code Generation with Vision Language Models for Robot arms application}
\author{Tran Quang Minh, Luu Trong Hieu, Nguyen Cong Khanh, Nguyen Quang Trung}
\institute[FPT University - VietDynamic JSC]{
 \textit{Department of Artificial Intelligence} \\
 FPT University - VietDynamic JSC}  % Fixed: Added missing closing brace
\date{Weekly Meeting}  % Fixed: Removed the broken syntax
% \logo{\includegraphics[width=.05\linewidth]{IU}}  % Commented out since IU image doesn't exist

%% Page number
\expandafter\def\expandafter\insertshorttitle\expandafter{%
 \insertshorttitle\hfill%
 \insertframenumber\,/\,\inserttotalframenumber}

\begin{document}

%% Title frame
\begin{frame}
    \titlepage
    % Removed the broken \item commands that were outside itemize environment
    \note{}
\end{frame}

%% ToC frame
\begin{frame}{Table of Contents}
    \tableofcontents
    % Removed the broken \item commands that were outside itemize environment
    \note{}
\end{frame}

\section{Introduction} 

%% TODO: Make sure to use \alert{} for highlighting keywords, and \cite{} to cite the corresponding quotations
\begin{frame}{Motivation}
    \begin{itemize} 	
        \item This is the first \alert{highlighted keyword} to emphasize an important concept.
        % \item The second point addresses \alert{another key idea} in \cite{knuth:1984}.  % Commented out citation
    \end{itemize}
    %% TODO: You can add the note here
    \note{}
\end{frame}

%% TODO: If you need a button, label the frames accordingly then using \hyperlink{label_of_the_dest_frame}{\beamerbutton{Name of the dest. frame}
\begin{frame}[label=objectives]{Objectives \hyperlink{scope}{\beamerbutton{Scope}}}
    \begin{block}{Sample Block Title}
        This block presents a \alert{key concept} that is crucial for understanding the topic.
    \end{block}
    \begin{alertblock}{Sample Alert Block Title}
        This block presents a more alarming \alert{key concept} that is crucial for understanding the topic.
    \end{alertblock}
    %% TODO: You can add the note here
    \note{}
\end{frame}

\begin{frame}{Actors \& Features}
    \textbf{Actors:}
                                    
    \textbf{Features:}
    %% TODO: You can add the note here
    \note{}
\end{frame}

\begin{frame}{Contributions}				
    \begin{block}{Scientific Contribution}
        % Add content here
    \end{block}						
    \begin{block}{Real-world Contribution}
        % Add content here
    \end{block}					
    %% TODO: You can add the note here
    \note{}
\end{frame}

\section{Related Work}

\begin{frame}{Research gaps}
    \begin{alertblock}{Research gap}
        % Add content here
    \end{alertblock}
                                                                
    $\Rightarrow$ \textbf{Concluding statement.}
                                                                
    %% TODO: You can add the note here
    \note{}
\end{frame}
         
\section{Proposed Method} 

%% TODO: Adjust the size of the figure by using [scale=x] or [widht=.x\linewidth] (x is a fraction) to fit within the frame. Then rename to your picture file name and add the caption
\begin{frame}{Overview}
    \begin{figure}
        \centering
        % \includegraphics[width=.8\linewidth]{samplel.png}  % Commented out since image doesn't exist
        \fbox{Figure placeholder - replace with your image}
        \caption{The caption of the figure.}
    \end{figure}	
    %% TODO: You can add the note here
    \note{}
\end{frame}

\begin{frame}[label=process1]{\texttt{Sample} Process \hyperlink{algo1}{\beamerbutton{Algorithm}} \hyperlink{pseudocode1}{\beamerbutton{Pseudocode}}}
    \begin{columns}
        \column{0.3\textwidth}
        \centering
        % \includegraphics[height=1.8\textwidth]{samplev.png}  % Commented out since image doesn't exist
        \fbox{Image placeholder}
        \column{0.7\textwidth}
        \begin{itemize}
            \item \textbf{Goal:} Add your goal here
            \item \textbf{Result:} Add your result here
            \item \textbf{Step:} Add your step here
            \item \textbf{Scope:} Add your scope here
        \end{itemize}
    \end{columns}
    %% TODO: You can add the note here
    \note{}
\end{frame}
                                
\section{Result} 
            
\begin{frame}{Prototyping}
    %% TODO: Add your GitHub repository link here
    \textbf{GitHub repository:} \url{https://github.com/Minhtrna/Code-gen-for-robot-arm-OJT-FALL-2025-FPT}
                                                                
    %% TODO: Add your demo website link here
    \textbf{Demo Website:} \url{https://example.com}
                            
    \begin{columns}
        \column{0.5\textwidth}
        \begin{figure}
            \centering
            % \includegraphics[width=.8\textwidth]{samples1.png}  % Commented out since image doesn't exist
            \fbox{Figure 1 placeholder}
            \caption{The caption of the figure.}
        \end{figure}	
        \column{0.5\textwidth}
        \begin{figure}
            \centering
            % \includegraphics[width=.8\textwidth]{samples2.png}  % Commented out since image doesn't exist
            \fbox{Figure 2 placeholder}
            \caption{The caption of the figure.}
        \end{figure}
    \end{columns}
    %% TODO: You can add the note here
    \note{}
\end{frame}
            
\section{Discussion} 
    
\begin{frame}{Limitations}
    \begin{itemize}
        \item Add your limitations here
    \end{itemize}
    $\Rightarrow$ \textbf{Concluding statement.}
    %% TODO: You can add the note here
    \note{}
\end{frame}

\begin{frame}{Comparison}
    \begin{table}[ht]
        \centering
        \caption{Comparison of different methods (\protect\cmark: YES, \protect\xmark: NO).}
        %% Comment the next line if the table width is relatively small
        \resizebox{\textwidth}{!}{%
            \begin{tabular}{lcccccc}
                \hline
                          & \textbf{Your Method} & Method B & Method C & Method D & Method E & Method F \\ \hline
                Feature 1 & \cmark               & \cmark   & \xmark   & \cmark   & \xmark   & \cmark   \\ 
                Feature 2 & \cmark               & \xmark   & \cmark   & \cmark   & \cmark   & \xmark   \\ 
                Feature 3 & \xmark               & \cmark   & \cmark   & \xmark   & \xmark   & \cmark   \\ 
                Feature 4 & \cmark               & \cmark   & \xmark   & \xmark   & \cmark   & \xmark   \\ 
                Feature 5 & \xmark               & \xmark   & \cmark   & \cmark   & \xmark   & \cmark   \\ 
                Feature 6 & \cmark               & \xmark   & \cmark   & \xmark   & \xmark   & \xmark   \\ \hline
            \end{tabular}%
        }
    \end{table}
    %% TODO: You can add the note here
    \note{}
\end{frame}
                                 	
\section{Conclusion}

\begin{frame}{Demonstration}
    \begin{block}{Process A}
        Add content here
    \end{block}
    \begin{block}{Scenario 1}
        Add content here
    \end{block}
    \begin{alertblock}{Scenario 2}
        Add content here
    \end{alertblock}
    \begin{block}{Process B}
        Add content here
    \end{block}
    %% TODO: You can add the note here
    \note{}
\end{frame}
                                             			
%% Thank You frame
\begin{frame}
    \centering
    % \includegraphics[width=.7\linewidth]{thankyou}  % Commented out since image doesn't exist
    \Huge Thank You!
    %% TODO: You can add the note here
    \note{}
\end{frame}
                    
%% Appendix frames
\appendix
                    
\begin{frame}[label=scope]{Scope \hyperlink{objectives}{\beamerbutton{Back to Objectives}}}
    Add scope content here
    %% TODO: You can add the note here
    \note{}
\end{frame}

\begin{frame}[label=algo1]{Formalizing - \texttt{Sample} Algorithm \hyperlink{process1}{\beamerbutton{Back to $\texttt{Sample}$ process}}}
    \begin{algorithm}[H]
        \small
        \caption{$(\text{Result}) \gets \texttt{Sample}(\text{Input1})$}
        \label{alg:algo1}
        \begin{algorithmic}[1]
            \Require $\text{Input1}$ is a predefined parameter.
            \State $\text{Set} \gets \emptyset$
            \For{$\text{element} \in \text{Input1}$}
            \If{$\text{Condition}(\text{element})$ is true}
            \State $\text{Set} \gets \text{Set} \cup \{\text{Process}(\text{element})\}$
            \Else
            \State \textbf{continue}
            \EndIf
            \EndFor
            \State $\text{Intermediate} \gets \texttt{Transform}(\text{Set})$
            \State \Return $\text{Result}$
        \end{algorithmic}
    \end{algorithm}
    %% TODO: You can add the note here
    \note{}
\end{frame}
    
\begin{frame}[label=pseudocode1]{Formalizing - \texttt{Sample} Pseudocode \hyperlink{process1}{\beamerbutton{Back to $\texttt{Sample}$ process}}}
    \begin{algorithm}[H]
        \small
        \caption{$(\text{Result}) \gets \texttt{Sample}(\text{Input1})$}
        \label{alg:pseudocode1}
        \begin{algorithmic}[1]
            \Require $\text{Input1}$ is a predefined parameter.
            \State $\text{Set} \gets \emptyset$
            \For{$\text{element} \in \text{Input1}$}
            \If{$\text{Condition}(\text{element})$ is true}
            \State $\text{Set} \gets \text{Set} \cup \{\text{Process}(\text{element})\}$
            \Else
            \State \textbf{continue}
            \EndIf
            \EndFor
            \State $\text{Intermediate} \gets \texttt{Transform}(\text{Set})$
            \State \Return $\text{Result}$
        \end{algorithmic}
    \end{algorithm}
    %% TODO: You can add the note here
    \note{}
\end{frame}

\begin{frame}[allowframebreaks, noframenumbering]{References}
    % \printbibliography[heading=none]  % Commented out since References.bib might be empty
    No references yet. Add them to References.bib file.
\end{frame}
                    
\end{document}